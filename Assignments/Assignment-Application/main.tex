\documentclass[12pt]{article}
\title{Ant colony optimization for traveling salesman}
\author{Ahmed Ashraf Mohamed \\ 2103134}
\date{}
\usepackage[
    left= 1in,
    right=1in,
    top=1in,
    bottom = 1in]{geometry}
    \usepackage{algorithm}
\usepackage{algpseudocode}
\begin{document}
\maketitle
\section{Introduction}
\subsection{Definition of ACO}
Ant algorithms are population-based approach to several problems, ant algorithms have been inspired by the behavior of real ant colonies, specifically their foraging behavior. The main idea of ant algorithms is the indirect communication of a colony of agents, artificial ants, based on pheromone trails.
\subsection{ACO applications}
\begin{enumerate}
    \item Traveling salesman
    \item Quadratic assignment
    \item Scheduling problems
    \item Vehicle routing
    \item Connection-oriented network routing
    \item Connection-less network routing
    \item Graph Coloring
\end{enumerate}
We have chosen \textbf{Traveling salesman problem} for this report.
\subsection{Definition of traveling salesman problem}
TSP is the problem of a salesman who wants to find, starting from his home town, a shortest possible trip through a given set of customer cities and to return to his home town. More formally, it can be represented by a completed weighted graph.
\section{The basic ACO for TSP}
In the ACO algorithms ants are simple agents which, in the TSP case, construct tours by moving from city to city on the problem graph. The ants' solution construction is guided the pheromone trails. When applying ACO algorithm to the TSP, a pheromone strength \(\tau_{ij} (t)\) is associated to each arc  \(\left(i,j\right)\), where \(\tau_{ij}(t)\) is a numerical information which is modified during the run of the algorithm and \(t\) is the iteration counter.

Initially, each of the \(n\) ants is placed on a randomly chosen city and then iteratively apples at each city a state transition rule. An ant constructs a tour as follows. At a city \(i\), the ant chooses a still unvisited city \(j\) probabilistically, biased by the pheromone trail strength \(\tau_{ij}(t)\) on the arc between city \(i\) and city \(j\) and a locally available heuristic information, which is a function of the arc length. Ants probabilistically prefer cities which are close and are connected by arcs with a high pheromone trail strength.

After all ants have constructed a tour, the pheromones are updated. This is done by first lowering the pheromone trail strengths by a constant factor and then the ants are allowed to deposit pheromone on the arcs they have visited. The trail update is done in such a form that arcs contained in shorter tours and/or visited by many ants receive a higher amount of pheromone and are therefore chosen with a higher probability in the following iteration of the algorithm. This way the amount of pheromone \(\tau_{ij}(t)\) represents the learned desirability of choosing next city \(j\) when an ant is at city \(i\)
\begin{algorithmic}[1]
    \Procedure{ACO algorithm for TSPs}{}
    \State Set parameters, initialize pheromone trails
    \While{termination condition not met}
    \State Construct Solutions
    \State Apply Local Search \Comment{optional}
    \State Update Trails
    \EndWhile
    \EndProcedure
\end{algorithmic}

\textit{Tour construction. } In ACS (Ant colony system) ants choose the next city using the \textit{pseudo-random-proportional action choice rule: } when located at city \(i\), ant \(k\) moves, with probability \(q_0\), to city \(l\) for which \(\tau_{il}(t)\) is maximal, that is, with probability \(q_0\) the best possible move as indicted by the pheromone trails.

\textit{Global pheromone trail update. } In ACS (Ant colony System) only the global best ant is allowed to add pheromone after each iteration. Thus, the update according to
\begin{equation}
    \tau_{ij}(t +1) = (1 - \rho) \cdot \tau_{ij}(t) + \rho \cdot \Delta \tau^{gb}_{ij}(t)
\end{equation}

\textit{Local pheromone trail update. } Additionally to the global updating rule, in ACS (Ant Colony System) the ants use a local update rule that they apply immediately after having crossed an arch during the tour construction:
\begin{equation}
    \tau_{ij} = (1 - \xi) \cdot \tau_{ij} + \xi \cdot \tau_0
\end{equation}

\end{document}