\documentclass{article}
\usepackage{booktabs}
\title{Examples of Evolutionary Computation Systems}
\author{}
\date{}
\begin{document}
\maketitle
\begin{center}
    {\large
        \begin{tabular}{l l}
            \toprule
            \textbf{Name}        & \textbf{ID} \\
            \midrule
            Ahmed Ashraf Mohamed & 2103134     \\
            \bottomrule
        \end{tabular}
    }
\end{center}
\section{Healthcare using systems equipped with Evolutionary Computation }
\subsection*{Human activity recognition (HAR)}

Human activity recognition (HAR) is considered one of the research fields related to
evolutionary computing where small but meaningful object detection plays an important role. This small object-detection activity is considered as one of the emerging
areas of image processing and machine vision too. The result of this field of study
helps to improve society by keeping a watch over the health of persons

\section{Finding The Lowest Energy Protein Folding Conformations In A 3D Space}

A sequence can fold in some directions depending on certain rules.
The objective here is to search for the folded sequences that adhere to the least energy states (The rules of folding are related to this).
In the second case, one can use EC to simulate mutations,insertions, and deletions in an amino acid sequence and hence track similarity between two sequences to gain more insight into the genetic evolutionary process.
The two sequences being compared in this case might be obtained from two similar organisms.
\end{document}